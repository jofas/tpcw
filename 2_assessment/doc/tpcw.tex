\documentclass[twoside,11pt]{article}
\PassOptionsToPackage{hyphens}{url}
\usepackage{jmlr2e}
\usepackage{amsmath}
\usepackage[toc,page]{appendix}
\usepackage[table]{xcolor}
\usepackage[marginparsep=30pt]{geometry}
\usepackage{stmaryrd}
\usepackage{algorithm}
\usepackage{algorithmic}
\usepackage{tikz}
\usepackage{pgfplots}
\usepackage{tabu}
\usepackage{longtable}
\usepackage{tabularx}
\usepackage{listings}
\usepackage{fancyref}
\usepackage{relsize}
\usepackage{float}
\usepackage{subcaption}
\usepackage{diagbox}

\usetikzlibrary{%
    arrows,
    arrows.meta,
    decorations,
    backgrounds,
    positioning,
    fit,
    petri,
    shadows,
    datavisualization.formats.functions,
    calc,
    shapes,
    shapes.multipart,
    matrix,
    plotmarks
}

\usepgfplotslibrary{fillbetween, statistics}

\pgfplotsset{
  compat=1.3,
  every non boxed x axis/.style={
  enlarge x limits=false,
  x axis line style={}%-stealth},
  },
  every boxed x axis/.style={},
  every non boxed y axis/.style={
  enlarge y limits=false,
  y axis line style={}%-stealth},
  },
  every boxed y axis/.style={},
}

\def\titl{Threaded Programming coursework II: the affinity
  schedule for scheduling the OpenMP loop construct}

\title{\titl}

\author{}

\ShortHeadings{B160509}{B160509}
\firstpageno{1}


\begin{document}

\maketitle

\begin{abstract}
\end{abstract}

\begin{keywords}
Scientific programming, parallelization,
performance optimization, OpenMP
\end{keywords}

\section{Introduction} % {{{

OpenMP version 4.5 supports various scheduling options for
its loop construct, for example \texttt{static},
\texttt{dynamic} or \texttt{guided}
\citep[see][Chapter 2]{omp}.
This paper presents an alternative schedule, called the
affinity schedule.
The affinity schedule combines some properties of the three
above mentioned scheduling options into one schedule.

This paper is a follow-up of a benchmark presented in
\citet{b1}.
\citet{b1} presents a scientific program written in the
Fortran programming language, containing two loops
performing matrix and vector operations.
These two loops were parallelized using built-in scheduling
options of OpenMP version 4.5 and then benchmarked in order
to determine the best schedule for the two loops.
The here presented affinity schedule is benchmarked the
same way, as are the built-in schedules in \citet{b1}.

This paper begins by describing two versions of the
affinity schedule. Afterwards the benchmark is described
and its results are presented. The benchmark of the
affinity schedule is then compared to the best schedule for
both loops determined in \citet{b1}.
At last the results are discussed and a conclusion is
drawn.

% }}}

\section{Method} % {{{

% }}}

\section{Results} % {{{

% }}}

\section{Discussion} % {{{

% }}}

\section{Conclusion} % {{{

% }}}

\bibliography{tpcw.bib}

\end{document}
